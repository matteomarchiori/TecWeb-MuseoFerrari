\section{Accessibilità}
\subsection{Colori}


\subsection{Navigazione nel sito}
La navigazione nel sito è garantita, in primo luogo, da una semplice \textit{navbar} che non include sottomenù. Sono direttamente visibili e accessibili tutte le pagine del sito web:
\begin{figure}[h]
		\begin{center}
		\includegraphics[scale=1.312]{Images/headerSito.png}
		\caption{Header del sito, in particolare visto dalla pagina mostre.}
	\end{center}
\end{figure}\\
Inoltre, il posizionamento corrente all'interno del sito web è chiarito grazie all'evidenziazione della pagina corrente all'interno della navbar:
\begin{figure}[h]
	\begin{center}
		\includegraphics[scale=1.312]{Images/selezionePaginaCorrente.png}
		\caption{Dettaglio: la pagina corrente è evidenziata rispetto alle altre.}
	\end{center}
\end{figure}\\
Oltre a ciò, sotto l'intestazione del sito è presente l'indicazione testuale sulla corrente posizione all'interno del sito ("\textit{breadcrumbs}"):
\begin{figure}[h]
	\begin{center}
		\includegraphics[scale=0.6]{Images/breadcrumbs.png}
		\caption{Dettaglio: \textit{breadcrumbs}.}
	\end{center}
\end{figure}\\
Per quanto concerne i link presenti nel sito, è stato fatto in modo che link "\textit{visitati}" e "ancora da visitare" siano facilmente distinguibili. Un esempio è visibile nella figura \ref{fig:linkVisitatiDaVisitare}.
\begin{figure}[h]
	\begin{center}
		\includegraphics[scale=1.5]{Images/linkVisitatiDaVisitare.png}
		\caption{Dettaglio: link da visitare e visitati. Il link centrale è stato già visitato mentre il primo e l'ultimo risultano ancora da esplorare.}
		\label{fig:linkVisitatiDaVisitare}
	\end{center}
\end{figure}\\
Per rendere efficiente la navigazione all'interno del sito anche ad utenti con difficoltà visive, è stato fatto uso di \textit{tabindex} per consentire un'agile navigazione all'interno del sito. Ogni immagine è inoltre accompagnata da un \textit{alt} che descrive in modo soddisfacente l'immagine in oggetto.\\
È stato inoltre inserito un bottone che consente di tornare facilmente in testa alla pagina.

\subsection{Mobile}
\subsection{Test accessibilità}
Con lo scopo di verificare e accertarsi che i livelli di accessibilità raggiunti fossero soddisfacenti abbiamo effettuato vari test. Abbiamo eseguito questi test consapevoli che non avrebbero mai potuto produrre un risultato esaustivo sul livello generale di accessibilità del sito.\\
Nei prossimi paragrafi mostriamo, con l'ausilio di alcuni screenshots, alcuni tra i risultati ottenuti.
\subsubsection{Test daltonismo}

\begin{figure}[!h]
	\begin{center}
		\includegraphics[scale=0.144]{Images/original.png}
		\includegraphics[scale=0.6]{Images/protanopia.png}
		\caption{Test daltonismo protanopia}
	\end{center}
\end{figure}
\begin{figure}[!h]
	\begin{center}
		\includegraphics[scale=0.144]{Images/original.png}
		\includegraphics[scale=0.6]{Images/deuteranopia.png}
		\caption{Test daltonismo deuteranopia}
	\end{center}
\end{figure}
\begin{figure}[!h]
	\begin{center}
		\includegraphics[scale=0.144]{Images/original.png}
		\includegraphics[scale=0.6]{Images/tritanopia.png}
		\caption{Test daltonismo tritanopia}
	\end{center}
\end{figure}
