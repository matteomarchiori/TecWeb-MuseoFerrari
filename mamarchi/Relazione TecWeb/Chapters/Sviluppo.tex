\section{Sviluppo}

\subsection{Progettazione}
Nella fase iniziale, grazie ai confronti avvenuti in alcune riunioni, il gruppo ha definito le linee guida da seguire nello svolgimento del progetto. Sono state scelte le tecnologie da utilizzare e sono stati divisi i compiti, nel modo più equo possibile, tra i membri del gruppo. È stato scelto di non utilizzare framework, ma di implementare la presentazione esclusivamente tramite CSS puro e il comportamento tramite JS puro. Per facilitare la realizzazione di pagine responsive, è stato scelto di utilizzare un layout grafico righe/colonne, implementato in CSS. 
\\Nella prima fase lo sviluppo del sito si è concentrato sul creare la struttura delle pagine in HTML e nell'aggiungere ad esse le regole di stile, creando prima il file \textit{style.css} e poi \textit{style\_desktop.css}, secondo la logica mobile first; subito dopo è stato aggiunto anche lo stile per la stampa. In seconda fase è stata aggiunta tutta la struttura dinamica del sito ed è stato quindi creato anche il database. Infine, in una terza fase, sono state implementate tutte le accortezze per garantire l'accessibilità del sito e sono state svolte le attività di verifica della qualità del sito. La parte Javascript del sito è stata realizzata nella prima e nella terza fase di sviluppo del sito. 
\\Riteniamo opportuno precisare che è stato scelto di evitare l'implementazione della parte di gestione dei contenuti del sito, caricamento di articoli immagini ecc. Questa parte sarebbe stata essenziale nella realtà, se il sito fosse stato realizzato per essere venduto ad un qualsiasi cliente, ma essendo un progetto puramente a scopo accademico, abbiamo evitato di realizzare tale parte del sito sostanzialmente per non aggiungere inutile complessità ad esso e soprattutto per motivi di tempo.

\subsection{Design}\label{design}
Per quanto riguarda i principi di design, il gruppo ha cercato di mantenere l'aspetto il più possibile semplice, pulito e ordinato. Il sito presenta un layout per desktop ed un altro per i dispositivi mobile. Il primo è stato pensato per mantenere un aspetto standard per tutte le tipologie di schermo, infatti è stato scelto di limitare la larghezza della sezione principale in modo da rendere piacevole la lettura anche agli utenti con schermi dal rapporto larghezza/altezza più alti. Il layout mobile, invece, è stato progettato per raggiungere una larghezza minima di 264px, a queste condizioni restrittive il menù principale viene nascosto e diventa utilizzabile attraverso un "\textit{burger}" \textit{button}: il contenuto viene disposto in un unica colonna prestando particolare attenzione a mantenere la giusta dimensione del testo.\\
Il contenuto di tutte le pagine è stato strutturato in modo da fornire solo le informazioni strettamente necessarie allo scopo della pagina stessa, cercando di disporle nella maniera più comprensibile possibile seguendo costantemente un ordinamento basato sull'importanza dell'informazione rispetto all'argomento relativo alla pagina. È stato scelto un menù di navigazione ad un singolo livello cercando anche di minimizzare l'ampiezza per non causare disorientamento agli utenti, rispettando la cosiddetta regola \textit{"dal particolare al generale"}.\\
Sia per il layout desktop, che per il layout mobile, sono stati selezionati pochi colori prestando attenzione a disporli con lo scopo di creare un adeguato contrasto in modo da facilitare la lettura a tutti gli utenti. Durante tutta la fase di progettazione e realizzazione del sito uno degli obiettivi più importanti per il gruppo è stato quello di raggiungere il più alto livello di accessibilità possibile.

\subsection{HTML e CSS}
La struttura del sito è stata creata adottando il linguaggio XHTML per garantire una compatibilità stabile e consolidata con tutti i browser e minimizzare il rischio di possibili problemi di accessibilità dati da HTML5.\\
La completa separazione tra struttura e presentazione è stata raggiunta utilizzando tre diversi fogli di stile: uno per i dispositivi mobili, uno per il desktop ed uno per la stampa delle pagine del sito.\\
 Abbiamo adottato un strategia mobile first realizzando quindi prima l'aspetto mobile e integrando in seguito il layout desktop. Per questo motivo il primo foglio di stile relativo al layout mobile, \textit{style.css}, è quello il più ricco, mentre \textit{style\_desktop.css} contiene solo regole per adattare il sito ad un layout desktop. In entrambi i layout abbiamo utilizzato uno schema a colonne per disporre il contenuto nelle pagine. Si è cercato dunque di minimizzare il numero di colonne per non complicare l'aspetto generale del sito che vuole essere il più lineare e semplice. Per ottenere una disposizione migliore di alcuni elementi, il sistema prevede che ogni contenitore occupante un certo numero di colonne potesse ospitare a sua volta altri elementi occupanti un certo numero di colonne, in modo ricorsivo.

\subsection{MySQL e PHP}
Il lato server è una parte molto importante e consiste nell'elaborazione dei dati che saranno poi inviati al client. Il database che viene utilizzato è stato sviluppato per la gestione delle seguenti informazioni:
\begin{itemize}
	\item auto esposte;
	\item biglietti;
	\item eventi;
	\item utenti.
\end{itemize}
Come DBMS abbiamo deciso di utilizzare MySQL e come engine di storage InnoDB che ci garantisce un buon compromesso tra dimensioni massime (64TB) e la sicurezza nel mantenere la consistenza dei dati.\\
Non è stato necessario costruire alcun trigger nel DBMS, non ne è stata vista la necessità data la semplicità del sito implementato.

\subsection{JavaScript}
Particolare attenzione è stata data a JavaScript, esso è linguaggio di scripting di tipo client side che permette una	maggiore accessibilità e dinamismo. È stato utilizzato per compiti rilevanti ma non fondamentali. Tale scelta trova giustificazione nel fatto che il sito deve restare usufruibile anche nel caso in cui venga consultato da dispositivi che non supportano JavaScript, o che hanno disabilitato gli script lato client.\\
Pertanto i compiti che gli sono affidati sono i seguenti:
\begin{itemize}
	\item gestione dinamica della navbar ad "\textit{hamburger}";
	\item controllo dei dati in input nei form della pagina "\textit{Biglietti}";
	\item controllo dei dati in input nei form della pagina "\textit{Info e Contatti}".
\end{itemize}
Per gli ultimi due punti, il controllo dei dati non è unico: viene effettuato anche lato server attraverso controlli in PHP, per garantire la correttezza dei dati qualora JavaScript fosse disabilitato.
