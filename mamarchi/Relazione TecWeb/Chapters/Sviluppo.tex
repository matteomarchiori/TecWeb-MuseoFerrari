\section{Sviluppo}

\subsection{Progettazione}

\subsection{Design}

\subsection{HTML e CSS}

\subsection{MySQL e PHP}

\subsection{JavaScript}
All'interno del progetto JavaScript è stato utilizzato per compiti rilevanti ma non fondamentali. Tale scelta trova giustificazione nel fatto che il sito deve restare usufruibile anche nel caso in cui venga consultato da dispositivi che non supportano JavaScript, o che hanno disabilitato gli script lato client.\\
Pertanto i compiti che gli sono affidati sono i sequenti:
\begin{itemize}
	\item gestione dinamica della navbar ad "\textit{hamburger}";
	\item controllo dei dati in input nei form della pagina "\textit{Biglietti}";
	\item controllo dei dati in input nei form della pagina "\textit{Info e Contatti}".
\end{itemize}
Per gli ultimi due punti, il controllo dei dati non è unico: viene effettuato anche lato server attraverso controlli in PHP, per garantire la correttezza dei dati qualora JavaScript fosse disabilitato.
